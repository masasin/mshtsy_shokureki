\documentclass[10pt, a4paper]{article}
\usepackage{resume_template_jp}
\usepackage{hyperref}
\usepackage{xeCJK}
\usepackage{topcapt,booktabs,multirow}
\setCJKmainfont[BoldFont=Yu Mincho Extrabold]{YuMincho Medium}
\setCJKsansfont[BoldFont=Yu Gothic Bold]{YuGothic Medium}
\usepackage{dcolumn} % for apsrtable outputs
\usepackage{xunicode} % extra support for unicode
\XeTeXlinebreaklocale "ja"
\XeTeXlinebreakskip=0pt plus 1pt
\XeTeXlinebreakpenalty=0

% value > 0
\def\xeCJKembold{0.15}

% hack into xeCJK, you don't need to understand it
\def\saveCJKnode{\dimen255\lastkern}
\def\restoreCJKnode{\kern-\dimen255\kern\dimen255}

% save old definition of \CJKsymbol and \CJKpunctsymbol for CJK output
\let\CJKoldsymbol\CJKsymbol
\let\CJKoldpunctsymbol\CJKpunctsymbol

% apply pdf literal fake bold
\def\CJKfakeboldsymbol#1{%
\special{pdf:literal direct 2 Tr \xeCJKembold\space w}%
\CJKoldsymbol{#1}%
\saveCJKnode
\special{pdf:literal direct 0 Tr}%
\restoreCJKnode}
\def\CJKfakeboldpunctsymbol#1{%
\special{pdf:literal direct 2 Tr \xeCJKembold\space w}%
\CJKoldpunctsymbol{#1}%
\saveCJKnode
\special{pdf:literal direct 0 Tr}%
\restoreCJKnode}
\newcommand\CJKfakebold[1]{%
\let\CJKsymbol\CJKfakeboldsymbol
\let\CJKpunctsymbol\CJKfakeboldpunctsymbol
#1%
\let\CJKsymbol\CJKoldsymbol
\let\CJKpunctsymbol\CJKoldpunctsymbol}

\usepackage{datetime2}

\DTMnewdatestyle{jpdate}{%
  \renewcommand*{\DTMdisplaydate}[4]{##1年##2月##3日}%
  \renewcommand*{\DTMDisplaydate}{\DTMdisplaydate}%
}
\DTMsetdatestyle{jpdate}

\begin{document}
\thispagestyle{plain}
\makeheading[{職務経歴書}]{勝者神優{\scriptsize{マサヒト・シンユウ (工学修士)}}}

%\section{\footnotesize Github: \href{https://github.com/masasin}{masasin}}
%\section{\footnotesize LinkedIn: \href{https://linkedin.com/in/masasin/?locale=ja_JP}{masasin}}
%\section{\footnotesize{\today 現在}}
\hspace{-3cm}{\footnotesize masasin (%
    \href{https://github.com/masasin}{Github}, 
    \href{https://linkedin.com/in/masasin/?locale=ja_JP}{LinkedIn}%
)}
\hfill{\footnotesize \today 現在}

\section{自己PR}
慎重に働くロボットやPythonの専門家である.
様々な職種に携わってきたおかげで,新しい事を早く習得し,幅広い知識とスキルや,多角的な視野を身に付けることができる.
更に,英語,フランス語,日本語等を含めて,様々な言語を話せる.
国や言葉や文化や職種にとって適応性が高い.

%様々な職種に携わってきたおかげで,幅広い知識とスキルや,多角的な視野を身に付けることができた.
% \\

%前職と大学でのフルタイム有償インターンシップを合わせて,約3年間働いて,様々な業界(建築・ロボット・組み込み・制御・電気・自転車・機械設計・産業・プログラミング・データサイエンス等々)で実務経験を得た.
%それに加えて,大学や大学院でロボットに関する専門知識を取得して,リーダーシップスキルも活かせた.
% \\

%更に,多言語を話せて,母国語は英語やフランス語を含めており,業務経験を4ヶ国で得たため,国や言葉や文化や職種にとって適応性が高い.
%今も勉強や取り組みを続けており,前向きで自己改善を常に目指している.
% 京都大学で修士号取得前、学部でロボット工学を学び,卒業するために6個のCo-op(有償インターンシップ)を修了することが必要でした.
% 各インターンシップは4ヶ月で,私の場合では3ヶ国で計2年間を働きました.
% インターンの労働時間・役割・収入・責任等は正社員のと比べて同じ位です.


\section{業務経験}
\begin{outerlist}
\item[\href{https://kapernikov.com}{\parbox[t]{3cm}{\raggedleft Kapernikov有限会社}}]{インダクタ4.0コンサルタント} \hfill {2019年3月〜2019年10月}

\hfill ベルギー,ブリュッセル
    \begin{innerlist}
    \item コンベヤベルトの障害物検出・警告システムの開発
    \begin{itemize}
      \item 下流の障害物になり得る物体をリアルタイムで検出した
      \item レーザープロファイラーを用いてPython 3やROSで開発した
      \item コンピュータービジョンでデータ解析を行った
      \item 生や解析したデータを可視化して,顧客のビデオ管理システムに送信した
      \item 現存のC++コードと連動しながらバグ修正を行った
    \end{itemize}
    \end{innerlist}

\item[\href{https://sentiance.com}{\parbox[t]{3cm}{\raggedleft Sentiance有限会社}}]{データサイエンティスト・Python専門家} \hfill {2018年2月〜2018年12月}

\hfill ベルギー,アントワープ
    \begin{innerlist}
    \item 新機能の導入や同僚の生産性の上昇化
    \begin{itemize}
        \item 会社のコードベースをPython 2からPython 3に移行していた
        \item コア機能をモジュール化された要素にリファクタリングした
        \item NumpyやScikit-learn等の機械学習モデルのビルドや検証を行った
        \item Spark等を利用し,新機能を導入して,プログラムを効率よくした
        \item DockerやPipenv等を使って,同僚の開発者の生産性を高めた
    \end{itemize}
    \item DevPI,Jenkins,Kafka,AWS等の知識を得た
    \item 開発はアジャイル手法を使用した
    \end{innerlist}
\end{outerlist}

\section{学術的経験}

\begin{outerlist}
\item[{\parbox[t]{3cm}{\raggedleft 京都大学\\松野研究室\\(メカトロニクス)}}]{修士課程} \hfill {2014年4月〜2017年3月}

\vspace{-2\baselineskip}
\hfill 京都府京都市
  \begin{innerlist}
  \item PythonやROSを使用し,過去画像を用いたドローンの遠隔操作システムを開発した
  \item ロボカップジャパンで参加した遠隔救命ロボットのソフトウェア開発チームを導いた
  \end{innerlist}

\item[{\parbox[t]{3cm}{\raggedleft ウォータールー大学}}]{メカトロニクス工学専攻} \hfill {2008年9月〜2013年7月}

\hfill カナダ,オンタリオ州,ウォータールー
  \begin{innerlist}
  \item 大学の工学協会のメンヘラ意識委員会を運営し,工学大使制度を共同で創設した
  \item 学部生バイオエンジニアの秘書として,国際シンポジウムを企画・開催した
  \item 様々なロボット関連活動を行い,機械やロボットの操作・制御・設計・開発等もした
  \end{innerlist}

\end{outerlist}

\section{スキル}
下記の項目は実務経験,又は教育や個人プロジェクトで使用経験があります.
\begin{outerlist}
\item[プログラミング言語] Python (Numpy, Scipy, Pandas, Sklearn, 視覚化等も), ROS, C++, C, Matlab, Bash等
\item[ソフト開発技術] Git, PyPI, Pipenv, Docker, Spark, Kafka, Devpi, Tox, Travis, Jenkins, AWS
\item[OS] Linux (Arch, Fedora, Gentoo, Ubuntu等), Mac OS X, Windows (XP〜10)
\item[環境] Raspberry Pi, Arduino, ESP8266, MicroPython, mbed, PIC, AVR, Allen Bradley PLC
\item[3Dソフト] Solidworks, Autodesk Inventor, AutoCAD, Sketchup
\item[その他] Vim, Gimp, Inkscape, \LaTeX, LibreOffice, Microsoft Office, Google Docs
\end{outerlist}

\section{言語}
\begin{lonelist}
\item[ネイティブ] 英語,フランス語,レバノン語
\item[流暢] 日本語,アラビア語
\item[中級] オランダ語,スペイン語
\item[初級] ドイツ語,クリオ語,その他
\end{lonelist}

\section{資格}
\begin{lonelist}
\item[2018年7月] 普通自動車 第一種免許 取得 (ベルギー,日本で切替可能)
\item[2017年3月] 日本機械学会 正員 入会
\item[2013年6月] 技術士研修(Engineer in Training)資格 取得
\item[2012年1月] 応急処置 心肺蘇生 (CPR-HCP) 資格 取得
\end{lonelist}

\section{業務経験\\(インターンシップ)}
ウォータールー大学のCo-operative Education (共同教育) 制度で,6ヶ所で4ヶ月ずつのフルタイム有償インターンシップが行う.
カナダの技術士組織に実務経験として認められる.

\begin{outerlist}
\item[\href{http://starquip.com/}{\parbox[t]{3cm}{\raggedleft Starquip Integrated Systems有限会社}}]{デザイン技術者} \hfill {2012年5月〜2012年8月}

\vspace{-\baselineskip}    
\hfill カナダ,オンタリオ州,トロント

  \begin{innerlist}
  \item カスタムな工業用の空気駆動・真空式のリフトアシスト装置を設計した
  \begin{itemize}
    \item Solidworksを使用して,お客様の個々のニーズに基づいて装置を設計した
    \item クライアントと共にデザインの正しさや実用性を確認しつつ開発した
  \end{itemize}
  \item Solidworksで,現存の製品の二次元図面を三次元化した
  \begin{itemize}
    % \item Solidworksのパーツやアセンブリを作成した
  \item ASME規格に則った図面を作成した
  \end{itemize}
  \item SolidworksやAutoCADのモジュール化で設計時間を85\%短縮した
  % \begin{itemize}
    % \item 従来の機械設計と空気回路設計図作成のプロセスをSolidworksとAutoCADでモジュール化した
  % \end{itemize}
  \item 3週間の間,部長が年休を取ってたため,管理なしに効率的に仕事が出来た
  \end{innerlist}

\item[\href{http://www.kqbikes.com/}{\parbox[t]{3cm}{\raggedleft Kevin Quan Studios有限会社}}]{ジュニアプロジェクト技術者} \hfill {2011年9月〜2011年12月}

\vspace{-\baselineskip}    
\hfill カナダ,オンタリオ州,トロント
  \begin{innerlist}
  \item マウンテンバイクや競走用の自転車の設計した
  \begin{itemize}
    % \item Solidworksで自転車のパーツとアセンブリを設計して図面を作成した
  \item 自転車のパーツ毎の金型を設計した
  \end{itemize}
  \item NACA 4桁系列に則った翼型生成プログラムを作成した
  % \begin{itemize}
    % \item Solidworksで使える翼型曲線を生成するPythonのプログラムを作成した
  % \end{itemize}
  \item 複数のロードバイクのそれぞれの空気抵抗を最小化した
  % \begin{itemize}
    % \item Solidworksで流体シミレーションを行った
    % \item 上記の翼型と自転車の断面を2Dと3Dで分析した
  % \end{itemize}
  \end{innerlist}

\item[\href{http://www.intellimec.com/}{\parbox[t]{3cm}{\raggedleft Intelligent Mechatronics Systems株式会社}}]{ハードウェア技術者} \hfill {2011年1月〜2011年4月}

\vspace{-2\baselineskip}
\hfill カナダ,オンタリオ州,ウォータールー
  \begin{innerlist}
  \item CやC++で,マイコンのプログラミングを行った
  \item Altium Designerを利用して,将来の商品の製作機を作成し,その商品をテストした
  \end{innerlist}

\item[\href{http://rpl.uwaterloo.ca/}{\parbox[t]{3cm}{\raggedleft ウォータールー大学\\マルチスケール\\積層造形研究室}}]{研究アシスタント} \hfill {2010年5月〜2010年9月}

\vspace{-2\baselineskip}
\hfill カナダ,オンタリオ州,ウォータールー
  \begin{innerlist}
  \item 超精密立体自由造形機の設計・見積もり・購入・加工・組み立て
  \item MatlabとOctaveを使って走査電子顕微鏡写真の画像処理を行った
  \end{innerlist}

\item[\href{http://www.aub.edu.lb/fea/me/research_labs/cvl/Pages/home.aspx}{\parbox[t]{3cm}{\raggedleft ベイルート・\\アメリカン大学\\移動ロボット研究室}}]{研究アシスタント} \hfill {2009年9月〜2009年12月}

\vspace{-2\baselineskip}
\hfill レバノン,ベイルート
  \begin{innerlist}
  \item 市販の慣性計測装置を使用して,C++でロボットの位置を推定した
  \item Matlabで多由来の位置推定を融合し,拡張カルマンフィルターで誤差を収めた
  \end{innerlist}

\item[{\parbox[t]{3cm}{\raggedleft Sierra Construction Systems有限会社}}]{訓練中エンジニア} \hfill {2009年1月〜2009年4月}

\vspace{-\baselineskip}
\hfill シエラレオネ,フリータウン
  \begin{innerlist}
  \item ExcelやVBAで,会社の支払い名簿やシステムの自動化し,人時効率を60倍向上した
  \item 建設プロジェクトの構造図・電気図を作成し,予算も算定した
  \end{innerlist}

\end{outerlist}

\flushright{以上.}

\end{document}
